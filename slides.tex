\documentclass[pdf]{beamer}
\usepackage{hyperref}
%\usepackage{listings}

\mode<presentation>{}

%% preamble

\title{Spark + R + RStudio}
%\subtitle{}
\author{\href{mailto:jeremy_j@ackso.net}{Jeremy Jackson}}

\begin{document}

\begin{frame}
	\titlepage
\end{frame}

%\begin{frame}{Frame Title}
%	Body of frame
%\end{frame}

\begin{frame}{Overview}
    \begin{itemize}
	\item Development environments and why you should use them.
	\item What is Spark?
	\item Setting up a local Spark + R + RStudio cluster
	\item SparkR and sparklyr 
	\item Scaling to production
    \end{itemize}
\end{frame}

\begin{frame}[fragile]{Resources}
    Apache Spark: \url{https://spark.apache.org}\\
    RStudio Server: \url{https://rstudio.com}\\
    RStudio sparklyr: \url{https://spark.rstudio.com}\\
    Jeremy's Github repo: \url{https://github.com/jeremy-j-ackso/sparkTest}\\

    \begin{verbatim}
	git clone https://github.com/jeremy-j-ackso/sparkTest.git
    \end{verbatim}
\end{frame}

\begin{frame}{Development environments and why you should use them}
    \begin{itemize}
	\item A virtual machine that is configured nearly identically to your deployment target.
	\item Might be small enough to live on your computer.
	\item Might be big enought to require a dedicated environment on a server or in a cluster.
	\item Causes everyone working on the project to have complete agreement on what is actually installed to the server. Or at least forces that discussion to occur.
    \end{itemize}
\end{frame}

\begin{frame}{Development environments and why you should use them}
    \begin{itemize}
	\item Makes it easier to catch errors.
	\item Makes it easier to catch misconfigurations.
	\item Lets you see how your code or model works in the context of other tenant systems.
	\item Configurations live in Version Control System.
    \end{itemize}
\end{frame}

\begin{frame}{What is Spark?}
    Spark is an in-memory parallel computing framework for manipulating and analyzing data.

\vspace{10pt}

    It is sponsored by DataBricks and the Apache Software Foundation.

\vspace{10pt}

    As of version 2.0 (released a few months ago) R is a fully-supported language, joining Python and Scala.

\vspace{10pt}

    Connectors exist for a large variety of databases: Hadoop, HBase, Cassanrda, MongoDB, ...

\vspace{10pt}

    It can also read a large number of file formats: ORC, Avro, Parquet, JSON, CSV, ....
\end{frame}

\begin{frame}{Setting up a local Spark + R + RStudio cluster}
    You need to have installed:
    \begin{itemize}
	\item Virtualbox
	\item Vagrant
    \end{itemize}

    It's a good idea (but not required) to have downloaded:
    \begin{itemize}
	\item The Ubuntu Trusty 64 (14.04) Vagrant Box
	\item Spark deb-formatted binaries
	\item RStudio Server deb-formatted binaries
    \end{itemize}

    If you don't already have these downloaded, you might not be able to follow along (they're some large downloads), but we can talk more about it later.
\end{frame}

\begin{frame}[fragile]{Setting up a local Spark + R + RStudio cluster}
    So what now?

    I've put together some handy code for you:
    \begin{verbatim}
	git clone https://github.com/jeremy-j-ackso/sparkTest.git
    \end{verbatim}

    What you get:
    \begin{itemize}
	\item The \LaTeX\ code for these slides
	\item A Vagrantfile which contains Virtualbox configuration information, plus calls to the following files as appropriate.
	\item A systemSetup.sh file, which Vagrant will run when you provision a single-node Spark virtual machine.
	\item A masterSetup.sh file, which Vagrant will run when you provision the master node of a Spark virtual machine cluster.
	\item A followerSetup.sh file, which Vagrant will run when you provision the follower node(s) of a Spark virtual machine cluster.
    \end{itemize}
\end{frame}

\begin{frame}{Setting up a local Spark + R + RStudio cluster}
    Let's jump to a shell and take a look around.
\end{frame}

\begin{frame}{SparkR and sparklyr}
    SparkR is the R package that comes with Spark.

    It provides an R API into Spark and allows access to everything in Spark.

\vspace{20pt}

    sparklyr is RStudio's dplyr-compliant meta-package over SparkR.

    sparklyr even includes additional functionality to install Spark locally on your machine in single-node mode. Good for test drives, bad if you're prototyping code that will need to be geared for your target environment.

    You can probably also connect it into a running Spark cluster, but I have not tried this out.

    It is not available on CRAN at the moment, so install it using \texttt{devtools::install\_github}
\end{frame}

\begin{frame}{SparkR and sparklyr}
    Let's login to the RStudio server running on our cluster and try them out.
\end{frame}

\begin{frame}{Scaling to production}
    You have some work to do first:
    \begin{itemize}
	\item Work with others on your team (like the Sysadmins) to review and improve the security of your system setup files.
	\item Modify your system setup files for the full scale that they will need be running at. For instance, adding all of the other IP addresses to the \texttt{/etc/hosts} file.
	\item Make sure that the integration tests are covering your code.
	\item One it's running and in production, you should be able to login to RStudio to run your ad hoc analyses and work with your scheduled R scripts.
    \end{itemize}
\end{frame}

\begin{frame}{Scaling to production}
    If you're working alone or on a small team, you might have a bit more leeway.
    \begin{itemize}
	\item Launch your cluster on a cloud hosting service (AWS, Azure, Google Cloud, Digitial Ocean, ...)
	\item Modify your configuration scripts for security and the scale of your cluster.
	\item Run your configuration scripts across your cluster to get them all working together.
	\item Login to RStudio Server and get to work!
    \end{itemize}
\end{frame}

\begin{frame}{Scaling to production}
    Many managed cloud services also provide Spark as part of their Hadoop distribution. You just have to configure a VM in the cluster you spin up to run R and RStudio, which you can pull out of your system setup shell files.
    \begin{itemize}
	\item AWS
	\item Azure
	\item Google Cloud
    \end{itemize}
\end{frame}

\begin{frame}{Thank you!}
Questions?
\end{frame}

\end{document}
